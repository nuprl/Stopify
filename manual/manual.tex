\documentclass[10pt]{book}
\usepackage{pervasives}


\begin{document}

\title{Stopify User's Manual}

\begin{titlepage}
\centering

%\vspace{4in}
\includegraphics[width=0.5\textwidth]{logo.png}\par\vspace{1cm}

{\Huge\bfseries Stopify}

\vspace{0.25in}

{\itshape version 0.1.0}

\vspace{0.5in}

{\Large\bfseries User's Manual}

\vspace{1in}

{\itshape by}

\vspace{0.25in}

Samuel Baxter, Arjun Guha, Shriram Krishnamurthi, Rachit Nigam, and Joe~Gibbs Politz
\vfill
\end{titlepage}


\tableofcontents

\chapter{Acknowledgements}

Stopify draws inspiration from several existing systems, including debug.js,
Doppio, Gambit.js, Pyret, scheme2js. Unwinder, and WeScheme. We thank Emery
Berger, Benjamin Lerner, Robert Powers, and John Vilk for insightful
discussions. We thank Caitlin Santone for designing the Stopify logo.


\chapter{Introduction}

Stopify is a JavaScript-to-JavaScript compiler that makes JavaScript a better
target language for high-level languages and web-based programming tools.
Stopify enhances JavaScript with debugging abstractions, blocking operations,
and support for long-running computations.

Suppose you have a compiler \emph{C} from language \emph{L} to JavaScript. You
can apply Stopify to the output of \emph{C} and leave \emph{C} almost entirely
unchanged. Stopify will provide the following features:

\begin{enumerate}

\item Stopify will support long-running \emph{L} programs without freezing the
browser tab. In particular, programs can access the DOM and are not limited to
Web Workers.

\item Stopify can pause or terminate an \emph{L} program, even if it is an
infinite loop.

\item Stopify can set breakpoints or single-step through the \emph{L} program,
if \emph{C} generates source maps.

\item Stopify can simulate an arbitrarily deep stack and proper tail calls.
This feature is necessary to run certain functional programs in the browser.

\item Stopify can simulate blocking operations on the web.

\end{enumerate}

In many cases, it is possible to ``blindly'' use Stopify by applying it to the
output of your compiler. However, Stopify will compile faster, produce faster
code, and support more features, if your compiler cooperates with Stopify in
particular ways. This manual will guide you through using Stopify with your own
compiler.

\paragraph{Warning} This manual is a work in progress. Many Stopify features
remain undocumented. We will preserve the interfaces documented here in
subsequent releases of Stopify.

\section{Quick Start}\label{quickstart}

\begin{figure}
\lstset{language=js}
\begin{lstlisting}
const elt = document.createElement("div");
document.body.appendChild(elt);

var i = 0;
var j = 0;
while(true) {
  if (i++ == 10000000) {
    j++;
    elt.innerText = "Still running ... " + (new Date());
    i = 0;
  }
}
\end{lstlisting}
\caption{This program will make any web browser crash.\label{trivial-periodic}}
\end{figure}

This section guides you through installing Stopify and applying it to a simple
JavaScript program that runs forever and periodically prints the current time
(\cref{trivial-periodic}). However, since the program never yields
control to the browser's event loop (e.g., using \texttt{setTimeout}), nothing
will appear on the page and the browser tab will eventually crash. Stopify will
make this program behave more naturally and actually show output.

\begin{enumerate}

\item Install the Stopify executable using NPM (or Yarn):
\lstset{language=console}
\begin{lstlisting}
npm install -g stopify
\end{lstlisting}

\item Save the program in \cref{trivial-periodic} to the file
 \texttt{input.js}:


\item Use the Stopify compiler to stopify the program:

\lstset{language=console}
\begin{lstlisting}
stopify input.js output.js
\end{lstlisting}

\item The Stopify installation includes a copy of the Stopify runtime system
(\texttt{stopify.bundle.js})
Look up the path using the following command:

\lstset{language=console}
\begin{lstlisting}
stopify-path stopify.bundle.js
\end{lstlisting}


\item Create a simple web page
that first loads the Stopify runtime system (i.e., the path produced
in the previous step) and then uses the runtime system to load the saved
file:
\lstset{language=console}
\begin{lstlisting}
var runner = stopify.stopify("output.js");
runner.run(() => console.log("done"));
\end{lstlisting}

\Cref{trivial-driver} shows such a page.

\end{enumerate}

\begin{figure}
\lstset{language=console}
\begin{lstlisting}
<html>
  <body>
    <script src="stopify.bundle.js"></script>
    <script>
    var runner = stopify.stopify("output.js");
    runner.run(() => console.log("done"));
    </script>
  </body>
</html>
\end{lstlisting}
\caption{A basic driver to run a stopified program.\label{trivial-driver}}
\end{figure}
  

Finally, open the page in any browser. You'll find that the the program
``just works'' and periodically prints the current time. In contrast, if you
load \texttt{input.js} directly, it will not print anything and will eventually
crash the browser tab.

\chapter{Overview}

The Stopify package includes several components:

\begin{itemize}

  \item A command-line compiler (the \texttt{stopify} executable).

  \item A compiler that can be used as a library from Node.

  \item A runtime system that is bundled for use in the browser. You
  can get the local path of this bundle using the following command:
  \lstset{language=console}
  \begin{lstlisting}
  stopify-path stopify.bundle.js
  \end{lstlisting}

  \item A compiler and runtime system that are bundled for the browser.
  You can get the local path of this bundle using the following command:
  \lstset{language=console}
  \begin{lstlisting}
  stopify-path stopify-full.bundle.js
  \end{lstlisting}
  This bundle has a lot more code than the runtime-only bundle. However, running Stopify in the
  browser is useful if your users have spotty Internet connections.

\end{itemize}

Note that \Cref{quickstart} presented the command-line compiler with the
runtime-only bundle. The rest of this manual presents other ways to use
Stopify.

\section{The Command-Line Compiler}
\lstset{language=console}

The Stopify CLI compiler requires the name of the input file and the name
of the output file. In addition, the compiler has several
optional flags:

\begin{lstlisting}
stopify [flags] input.js output.js
\end{lstlisting}

The optional flags are:
\begin{itemize}

  \item \lstinline|-t| or \lstinline|--transform| selects the continuation implementation
  to use (\cref{transformation}).

  \item \lstinline|-n| or \lstinline|--new| selects the encoding for constructors (\cref{new-method}).

  \item \lstinline|--eval| or \lstinline|--no-eval| determine if JavaScript's \emph{eval} is supported
  (see \cref{eval-flag}).

  \item \lstinline|--es| determines if JavaScript's implicit operations supported (\cref{implicit-ops-flag}).

  \item \lstinline|--hofs| determines if JavaScript's builtin higher-order functions supported (\cref{hofs-flag}).

  \item \lstinline|--js-args| determines how faithfully should Stopify support the \emph{arguments} object (\cref{arguments-flag}).

  \item \lstinline|--getters| or \lstinline|--no-getters| determine if should Stopify support getters and setters (\cref{getters-flag}).

  \item \lstinline|--debug|: should Stopify support single-step debugging (\cref{debug-flag}).
\end{itemize}

If a flag is not set, Stopify picks a default value that is documented
in the subsection for each flag. By default, Stopify is \emph{not} completely
faithful to the semantics of JavaScript. (Certain JavaScript features are
difficult to support and incur a high runtime cost.) Instead, Stopify's default
values work with a number of compilers that we've tested. By default,
Stopify does not support getters, setters, eval, builtin higher-order
functions, implicit operations, \lstinline|arguments|-object aliasing,
and single-stepping.
If you think you may need these features, you will
need to set their corresponding flags.

\lstset{language=js}
To load a compiled file in the browser, the Stopify runtime provides
the following function:
\begin{lstlisting}
stopify.stopify(url: string, opts?: RuntimeOpts): AsyncRun
\end{lstlisting}

Above, \lstinline|url|should be the address of the compiled file (i.e., hosted on
your web server) and \lstinline|url| is an optional runtime configuration
(\cref{runtime-config}). The result of this function is an \lstinline|AsyncRun|
object (\cref{asyncrun}).

\section{Runtime Configuration\label{runtime-config}}

The Stopify runtime system takes a dictionary of options with the
following type:

\lstset{language=js}
\begin{lstlisting}
interface RuntimeOpts {
  estimator?: "velocity" | "reservoir" | "exact" | "countdown",
  yieldInterval?: number /* must be greater than zero */,

  stackSize?: number /* must be greater than zero */
  restoreFrames?: number /* must be greater than zero */,
}
\end{lstlisting}

The first two options control how frequently Stopify yields control to the browser
(\lstinline|yieldInterval|) and the mechanism that it uses to determine
elapsed time (\lstinline|estimator|). The last two options tell Stopify how deep
the browser stack is and how many frames should be restored onto the JS stack
after a stack capture.

\subsection{Time estimator \texttt{(.estimator)} }\label{estimator}
By default, Stopify uses the \texttt{velocity} estimator that samples the current
time (using \lstinline|Date.now()|) and tries to yield every 100 milliseconds.
The \texttt{velocity} estimator dynamically measures the achieved yield interval
and adapts how frequently it yields accordingly. This mechanism is inexact, but
performs well. You can adjust the yield interval, but we do not recommend using
a value lower than 100.

The \texttt{reservoir} estimator samples the current time using
\emph{reservoir sampling} (i.e., the probability of resampling the current time
decreases as the program runs longer). This technique
is less robust than \texttt{velocity} to fluctuations in program behavior, but
still outperforms other methods. This usually has a lower runtime overhead than
\texttt{velocity}, but sacrifices accuracy. We recommend \texttt{velocity} for a
more general, nondeterministic estimator.

The \texttt{countdown} estimator yields after exactly $n$
yield points have passed. With this estimator, the \texttt{yieldInterval}
is interpreted as the value of $n$ and not a duration. We do not recommend
using this estimator in practice, since a good value of $n$ will depend on
platform performance and program characteristics that are very hard to
predict. However, it is useful for reproducing bugs in Stopify, since
the \texttt{velocity} estimator is nondeterministic.

Finally, the \texttt{exact} estimator checks the current time at every
yield point, instead of sampling the time. This has a higher runtime
overhead than \texttt{velocity} and we do not recommend it.

\subsection{Stack behavior \texttt{(.stackSize and .restoreFrames)} }
Stopify supports heap bounded stacks for programming languages that want to
support deeply recursive code and not inherit the stack size limit imposed
by JavaScript VMs. Stopify accomplishes by restoring only
a suffix of a captured stack and saving the rest of it for the next restore
event. Once a program finishes running, Stopify checks to see if
there is a prefix of the stack that was previously saved and restores such a
prefix.
\\ \\
\texttt{.stackSize}: Controls the maximum number of stack frames that a stopified
program can consume before it must suspend. A constant value implies support for heap
bounded stacks. Defaults to \lstinline|Infinity|.
\\ \\
\texttt{.restoreFrames}: Controls the maximum number of stack frames that are
restored onto the JS stack after a capture event. The rest of the stack is saved
in the runtime. A constant value implies support for deep stacks. Cannot be
greater than \texttt{.stacksize}. Defaults to \lstinline|Infinity|.


\section{The \texttt{AsyncRun} Interface\label{asyncrun}}
\lstset{language=js}

\begin{figure}
\lstset{language=js}
\begin{lstlisting}
interface AsyncRun {
  run(onDone: () => void, onYield?: () => void, onBreakpoint?: (line: number) => void): void;
  pause(onPaused: (line?: number) => void): void;
  resume(): void;
  setBreakpoints(line: number[]): void;
  step(onStep: (line: number) => void): void;
  pauseImmediate(callback: () => void): void;
  continueImmediate(result: any): void;
}
\end{lstlisting}
\caption{The \texttt{AsyncRun} interface.}
\label{asyncrun-interface}
\end{figure}

The \lstinline|AsyncRun| interface (\cref{asyncrun-interface}) provides methods
to run, stop, and control the execution of a stopified program. The
interface provides several methods, none of which should be used
directly by the stopified program. The following methods are meant to be
used by the driver program that controls execution (e.g., a web-based IDE):

\begin{itemize}

  \item The \lstinline|run| method starts execution and requires a callback that
  gets invokes when execution completes. You may provide optional callbacks
  that are invoked when the program yields control and when a breakpoint is reached.

  \item The \lstinline|setBreakpoint| method sets the active breakpoints.

  \item The \lstinline|pause| method pauses the program at the next yield point
  and requires an optional callback that is invoked when the program has paused.

  \item The \lstinline|resume| method resumes execution after a pause.

  \item The \lstinline|step| method resumes execution and pauses again at the
  next yield point.

\end{itemize}

The following methods are are meant to be used by non-blocking
JavaScript functions to provide simulated blocking interface to the stopified
program:
\begin{itemize}

  \item The \lstinline|pauseImmediate| method suspends the stopified program
  and invokes the provided callback. A function should not execute anything
  after invoking \lstinline|pauseImmediate|. Typically, a function that
  uses \lstinline|pauseImmediate| will use it in a \lstinline|return| statement.

  \item The \lstinline|continueImmediate| function resumes execution with
  the provided value.

\end{itemize}

\Cref{illustrative-examples} has several examples that use these methods
to implement simulated blocking operations.

\section{The Online Compiler}
\lstset{language=console}

The file \lstinline|stopify-full.bundle.js| packages the compiler and runtime
system for use the browser. You can get the local path of this bundle using the
following command:
\lstset{language=console}
\begin{lstlisting}
stopify-path stopify-full.bundle.js
\end{lstlisting}

This bundle exposes the following function:

\lstset{language=js}
\begin{lstlisting}
stopify.stopifyLocally(url: string, copts?: CompileOpts, ropts?: RuntimeOpts): AsyncRun
\end{lstlisting}

The optional \lstinline|CompilerOpts| is dictionary with the following type:

\begin{lstlisting}
interface CompilerOpts {
  getters?: boolean,
  debug?: boolean,
  captureMethod?: "lazy" | "retval" | "eager",
  newMethod?: "wrapper" | "direct",
  eval?: boolean,
  es?: "sane" | "es5",
  hofs: "builtin" | "fill",
  jsArgs?: "simple" | "faithful" | "full",
  externals?: string[]
}
\end{lstlisting}



\section{Compiler Configuration}

\lstset{language=js}

You can configure the Stopify compiler in several ways. Some of these
options only affect performance, whereas other options affect the
sub-language of JavaScript that the compiler targets.

\subsection{Transformation (\texttt{.captureMethod})\label{transformation}}

Stopify uses first-class continuations as a primitive to implement its
execution control features. Stopify can represent continuations in several
ways; the fastest approach depends on the application and the browser. The
valid options are \lstinline|"lazy"|, \lstinline|"retval"|, and
\lstinline|"eager"|. For most cases, we recommend using \lstinline|"lazy"|

\subsection{Constructor Encoding (\texttt{.newMethod})\label{new-method}}

Stopify implements two mechanisms to support suspending execution within the
dynamic extent of a constructor call.

\begin{itemize}

\item \lstinline|"wrapper"| desugars all \lstinline|new| expressions to ordinary
function calls, using \lstinline|Object.create|.

\item \lstinline|"direct"| preserves \lstinline|new| expressions,
 but instruments all functions to check if they are invoked as constructors,
 using \lstinline|new.target|.

\end{itemize}

The fastest approach depends on the browser.
We recommend using \lstinline|wrapper|.

\subsection{Eval Support (\texttt{.eval})\label{eval-flag}}

How should Stopify handle JavaScript's \lstinline|eval| function? By default,
this flag is \lstinline|false| and Stopify leaves \lstinline|eval| unchanged.
Since Stopify typically does not rename variables, using a stopfied program can
use \lstinline|eval|, but the evaluated code may lock-up the browser if it has
an infinite loop.

If set to \lstinline|true|, Stopify rewrites calls to JavaScript's
\lstinline|eval| function to invoke the Stopify compiler. (Note: Stopify does
\emph{not} rewrite \lstinline|new Function| and dynamically generated
\texttt{<script>} tags.) This allows Stopify to control execution in
dynamically generated code. Naturally, this requires the online compiler.
However, the feature incurs considerable overhead.


\subsection{Implicit Operations (\texttt{.es})\label{implicit-ops-flag}}

Stopify can suspend execution within user-written \lstinline|valueOf()| and
\lstinline|toString()| methods that JavaScript invokes implicitly.

For example, the following program is an infinite loop in JavaScript:

\begin{lstlisting}
var x = { toString: function() { while(true) { } } };
x + 1;
\end{lstlisting}

With the implicit operations flag is set to \lstinline|"es5"|, Stopify will be
able to gracefully suspend the program above. With the flag set to \lstinline|"sane"|,
Stopify will not be able to detect the the infinite loop. We have found that
most source language compilers do not rely on implicit operations, thus it is
usually safe to use \lstinline|"sane"|.

\subsection{Fidelity of \texttt{arguments} (\texttt{.jsArgs})\label{arguments-flag}}

The \lstinline|arguments| object makes it difficult for Stopify to resume
execution after suspension. Stopify supports \lstinline|arguments| in full, but
it also supports two simple special cases that improve performance.

\begin{itemize}

  \item Use \lstinline|"simple"| if the program (1) does not use \lstinline|arguments| to
    access declared formal arguments and (2) only reads additional
    arguments using the \lstinline|arguments| object.

  \item Use \lstinline|"faithful"| if the program (1) does not use \lstinline|arguments|
    to access declared formal arguments and (2) may read or write additional
    arguments using the \lstinline|arguments| object.

  \item Use \lstinline|"full"| for full support of JavaScript's \lstinline|arguments| object.

\end{itemize}

\subsection{Higher Order Functions (\texttt{.hofs}) \label{hofs-flag}}

Programs cannot use bulitin higher-order functions (e.g., \lstinline|.map|,
\lstinline|.filter|, etc.) with Stopify, since Stopify cannot instrument native
code. The \lstinline|.hofs| flag has two possible values:

\begin{itemize}

  \item Use \lstinline|"builtin"| if the program does not use any native
  higher-order functions.

  \item Use \lstinline|"fill"| to have Stopify rewrite programs that use native
    higher-order functions to use polyfills written in JavaScript.

\end{itemize}

\subsection{Getters and Setters (\texttt{.getters})\label{getters-flag}}

Programs that suspend execution within getters/setters incur a lot of overhead
with Stopify. The \lstinline|.getters| flag has two possible values:

\begin{itemize}

  \item Use \lstinline|true| to have Stopify instrument the program to
  support suspension within getters and setters.

  \item Use \lstinline|false| if the program does not use getters and setters.

\end{itemize}

\subsection{Single-stepping and Breakpointing (\texttt{.debug})\label{debug-flag}}

 Set \lstinline|.debug| to \lstinline|true| to enable
support for single-stepping and breakpointing. However, note that this requires
more instrumentation and slows the program down further.

\chapter{Illustrative Examples\label{illustrative-examples}}

\lstset{language=js}
This chapter presents several examples that showcase Stopify's features.

\section{A Blocking \texttt{sleep} function}

The browser does not have a blocking \lstinline|sleep| function.
However, we can use \lstinline|window.setTimeout| and Stopify to simulate
a blocking \lstinline|sleep| operation:
\begin{lstlisting}
function sleep(duration) {
  asyncRun.pauseImmediate(() => {
    window.setTimeout(() => asyncRun.continueImmediate(undefined), duration);
  });
}
\end{lstlisting}

In the code above, \lstinline|asyncRun| is an instance of \lstinline|AsyncRun|
(\cref{asyncrun}). Note that this function should be stopified itself and
needs to be declared as an external. \Cref{sleep-complete} shows a complete example
of a page that uses \lstinline|sleep|.

\begin{figure}
\lstinputlisting{examples/sleep.html}
\caption{This program runs forever and prints a period each second.}
\label{sleep-complete}
\end{figure}

\section{A Blocking \texttt{prompt} function}

\begin{figure}
\lstinputlisting{examples/prompt.html}
\caption{This program prompts the user for two inputs without modal dialog boxes.}
\label{prompt-complete}
\end{figure}

The \lstinline|prompt| and \lstinline|alert| functions that are built-in to
browsers are not the ideal way to receive input from the user. First, modal
dialog boxes are unattractive; second, a user can dismiss them; and finally, if
a page displays too many modal dialog boxes, the browser can give the user to
suppress all of them (\cref{prompt-complete}).

\appendix

\chapter{Release Notes}

\section{Stopify (working version)}

\begin{enumerate}

  \item Exposed the \texttt{velocity} estimator and set it as the default
  instead of \texttt{reservoir}. In our experiments, \texttt{velocity} performs
  better and has negligible overhead.

\end{enumerate}

\section{Stopify 0.1.0}

Initial release

\end{document}